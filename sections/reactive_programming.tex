Šiame skyriuje trumpai aprašomas deklaratyvus programavimas, pateikiami faktai, kodėl toliau nebus kalbama apie funkcinį reaktyvų programavimą. Toliau seka įvadas į reaktyvų programavimą, parodantis paprastą reaktyvaus programavimo pavyzdį, aprašomos išskirtinės reaktyviojo programavimo kalbos ypatybės, tada seka platesnis reaktyvaus programavimo aprašymas, minintis būdingus reaktyvaus programavimo bibliotekų bruožus. Skaitytojas supažindinamas su įvykių ir pranešimų valdomais programavimo stiliais, kuriuos aprašo reaktyvumo manifestas, ko pasekoje aprašomos dokumente apibūdinamos reaktyvios sistemos, jų bruožai ir sąvybės, reaktyvaus programavimo sąsaja su reaktyviomis sistemomis. Galiausiai skyrius konkretizuoja reaktyvaus programavimo pranašumus bei parodo jo svarbą šio metu.

\subsection{Deklaratyvus programavimas}

Deklaratyvusis programavimas siekia paaiškinti, kokias reikšmes reikėtų apdoroti, o ne tai, kaip jas reikėtų apdoroti. Tiksliau tariant, jis apibūdina apdorojimo duomenų srautą, neapibūdindamas jo kontrolės srauto. Laikoma, kad deklaratyvusis stilius yra iš esmės pranašesnis už lygiavertį imperatyvųjį kodą \cite{DeclarativeProgramming}.

\subsection{Funkcinis reaktyvus programavimas}

Funkcinis reaktyvus programavimas, neretai vadinamas tiesiog FRP, dažnai yra nesuprastas. Prieš 20 metų Conal Elliott labai aiškiai apibrėžė FRP \cite{ElliottHudak97:Fran}. Daugiau šio mokslų daktaro darbų susijusių su FRP galima rasti jo asmeninėje svetainėje\footnote{http://conal.net/papers/frp.html}. ``Lambda Jam'' konferencijoje 2015 metais Conal Eliott teigia, jog FRP terminas buvo neteisingai panaudotas aprašant technologijas, tokias kaip Elm, Bacon.js, reaktyvius plėtinius (RxJava, Rx.NET, RxJS, RxRuby) bei kitas \cite{elliott:lambda:jam}. Dauguma bibliotekų, kurios teigia jog palaiko FRP, beveik vienareikšmiškai aprašo reaktyvų programavimą. Dėl šios priežasties, šiame darbe toliau bus kalbama tik apie reaktyvųjį programavimą.

\subsection{Įvadas į reaktyvųjį programavimą}

Reaktyvųjį programavimą galima laikyti tam tikra deklaratyviojo programavimo atmaina. Pagrindinė jo idėja – reaktyviųjų reikšmių samprata. Tokios reikšmės priklauso viena nuo kitos, jos dinamiškai keičiasi ir yra skaidriai atnaujinamos. RP tikslas – apibrėžti statinius arba dinaminius reikšmių ryšius, kai jų nereikia atnaujinti ar sinchronizuoti rankomis. Kodas tiesiogiai apibrėžia duomenų srautą, o kalba netiesiogiai apdoroja platinamus pakeitimus. \ref{reactive_pseudocode} kodo pavyzdyje pavaizduota reaktyviojo programavimo idėja imperatyvioje nuoseklių komandų aplinkoje. Reikšmė \lstinline|sum| nurodo skaičiavimą \lstinline|a + b|. Kai kuri nors iš šių reikšmių pasikeičia, \lstinline|sum| reikšmė išlieka teisinga.

\begin{lstlisting}[caption=Reaktyvaus programavimo pseudokodas, label=reactive_pseudocode]
a <- 0
b <- 5
sum <- a + b
a <- 40
print(sum) # Prints 45
\end{lstlisting}

Dinaminių kalbų atžvilgiu, RP galima apibrėžti kaip asinchroninį, kadangi visos reikšmės yra uždariniai visoje aplinkoje. Pavyzdžiui, programavimo kalboje „Ruby“ anksčiau pateiktą elgseną galima būtų sumodeliuoti naudojant \lstinline|lambda| funkciją be argumentų: \lstinline|sum =  -> { a + b }|. Pilnas pseudokodo realizacijos Ruby kalboje iliustracija pateikta \ref{reactive_ruby} kodo pavyzdyje.

\begin{lstlisting}[caption=Reaktyvaus programavimo pavyzdys Ruby kalboje, label=reactive_ruby]
a = 0
b = 5
sum = -> { a + b }
a = 40
p sum.call # Prints 45
\end{lstlisting}

Reaktyvumo koncepcija leidžia programuotojui reikšmes apibrėžti tik kartą ir nereikia iš naujo įvertinti išraiškos, kai pasikeičia dešinė aprašo pusė. Kai yra kompleksinių reikšmių, programuotojams nereikia tiesiogiai paskelbti funkcijos jai apskaičiuoti, taip pat nereikia perduoti teisingų argumentų iškvietimo metu.

\subsection{Išskirtinės reaktyviojo programavimo kalbos ypatybės}

Reaktyviojo programavimo kalbos suteikia (reaktyviąsias) reikšmes ir (reaktyviuosius) operatorius.

\subsubsection{Reaktyviosios reikšmės}

Reaktyviosios reikšmės yra įvykiai ir elgsenos \cite{Bainomugisha:2013:SRP:2501654.2501666}.

\begin{itemize}
  \item Įvykiai – reikšmė, kuri teikia begalinį keitimų srautą. Reikšmė pateikia keitimų įvykius, kurie gali turėti reikšmę kaskart, kai įvyksta keitimas. Kitaip nei elgsena (paaiškinta toliau), įvykių srautas neturi reikšmės.
  \item Elgsena – tai reikšmė, kuri keičiasi laike \cite{ElliottHudak97:Fran}. Apskritai ji reiškia funkcinę priklausomybę (elgsenos išraiška) nuo kitų elgsenų. Jos reikšmė gaunama vertinant elgsenos išraišką. Kai kuriose reaktyviojo programavimo kalbose elgsena dar vadinama signalu.
\end{itemize}

Daugelyje reaktyviojo programavimo kalbų galima naudoti įvykius ir elgsenas. Kai kuriose kalbose galima naudoti tik elgsenas, nes jas galima realizuoti kaip įvykių apibendrinimą \cite{Wan2002:EventDrivenFRP, CzaplickiC13:FRPGUI}. Šiose kalbose kūrėjas elgseną gali laikyti reikšmę turinčiu įvykiu.

Aprašant reaktyvųjį programavimą, daugelis elgsenos ir įvykių koncepcijų veikia vienodai. Be to, kalbos, kurios palaiko abi šias koncepcijas, leidžia operatoriams keisti įvykį į elgseną ir atvirkščiai. Todėl šiame darbe terminą „elgsena“ naudosime elgsenoms ir įvykiams aprašyti.

\subsubsection{Reaktyvieji operatoriai}

M.Viering išskiria reaktyvius operatorius kaip išskirtinę reaktyviojo programavimo ypatybę \cite{RubyReactiveMaster}. Reaktyviojo programavimo kalbos operatoriai veikia su įvykiais ir elgsenomis. Šie operatoriai gali transformuoti arba jungti elgsenas į naują elgseną. Reaktyviojo programavimo kalbos teikiami operatoriai skirtingose kalbose yra skirtingi. Tačiau paprastai yra operatorius, skirtas transformuoti elgseną (map), operatorius, skirtas sujungti elgsenas į vieną elgseną (merge), ir operatorius, kuris veikia esant būsenai (fold). Tai yra tipiniai daugelio reaktyviojo programavimo kalbų operatoriai, todėl toliau jie yra plačiau paaiškinami.

\begin{itemize}
  \item \textbf{map} - „map“ operatorius taiko funkciją \lstinline|f| elgsenai \lstinline|b|. Šio proceso metu sukuriama nauja elgsena ir šios naujos elgsenos reikšmė yra funkcija \lstinline|f|, taikoma elgsenos \lstinline|b| reikšmei.
  \item \textbf{merge} - „merge“ operatorius sujungia dvi elgsenas į naują elgseną. Naujos elgsenos reikšmė yra naujausios pakeistos elgsenos reikšmė.
  \item \textbf{fold} - „fold“ operatorius elgsenoje \lstinline|b| apima funkciją \lstinline|f| ir reikšmę \lstinline|v|. Jis sukuria naują elgseną \lstinline|b_new|. Pradinė \lstinline|b_new| reikšmė yra \lstinline|v|. Kai \lstinline|b| reikšmė pasikeičia, \lstinline|b_new| reikšmė pasikeičia į \lstinline|f| (sena \lstinline|b_new| reikšmė, \lstinline|b| reikšmė). Aprašant laisvai, \lstinline|b_new| reikšmė \lstinline|f| taikoma senai \lstinline|b_new| reikšmei ir \lstinline|b| reikšmei.
\end{itemize}

\subsection{Reaktyvus programavimas}

Reaktyvusis programavimas, kurio nereikėtų painioti su funkciniu reaktyviuoju programavimu, tai asinchroninio programavimo porūšis ir paradigma, kai logiką valdo atsiradusi nauja informacija, o ne kontrolės srautą valdo vykdymo gija.

Jis palaiko problemos suskaidymą į keletą atskirų veiksmų, kurių kiekvieną galima įvykdyti asinchroniniu ir neblokuojančiu būdu ir tada iš jų sudaryti darbo eigą, kurios įvestys ir išvestys gali būti begalinės.

„Oxford Dictionary“ žodį „asynchronous“\footnote{http://www.oxfordlearnersdictionaries.com/definition/english/asynchronous?q=asynchronous} apibrėžia kaip „nesantį ar nevykstantį tuo pačiu metu“, o tai šiame kontekste reiškia, kad pranešimo ar įvykio apdorojimas vyksta tam tikru atsitiktiniu laiku, tikėtina, ateityje. Tai labai svarbus reaktyviojo programavimo metodas, kadangi jis įgalina neblokuojantį vykdymą, kai vykdymo gijoms, konkuruojančioms dėl bendrai naudojamo ištekliaus, nereikia laukti dėl užblokavimo (dėl ko vykdymo gija negali atlikti kito darbo, kol neatliktas dabartinis) ir jos gali atlikti kitą naudingą darbą, kol išteklius yra užimtas. Amdalo dėsnis teigia \cite{Rodgers:1985:IMS:327070.327215}, kad konkurencija yra didžiausias išplečiamumo priešas, taigi reaktyvioji programa turėtų retai blokuoti arba neblokuoti visai.

Reaktyvųjį programavimą iš esmės valdo įvykiai, skirtingai nuo reaktyviųjų sistemų, kurias valdo pranešimai. Skirtumas tarp įvykių valdomo ir pranešimų valdomo programavimo bus paaiškintas tolesnėje šio darbo dalyje.

Reaktyviojo programavimo programų sąsajos (API) bibliotekos paprastai yra:

\begin{itemize}
  \item Atgalinių iškvietimų valdomos – anoniminiai pašalinio veikimo atgaliniai iškvietimai yra susieti su įvykių šaltiniais ir iškviečiami, kai įvykiai praeina pro duomenų srauto grandinę.
  \item Deklaratyvios – naudojamas funkcinis komponavimas ir paprastai nusistovėję kombinatoriai, pvz., perkoduoti, filtruoti, perlenkti ir kt.
\end{itemize}

Šį programavimo metodą palaikančių programavimo abstrakcijų pavyzdžiai:

\begin{itemize}
  \item Ateitis / pažadai (angl.: Futures / Promises) – atskiros reikšmės konteineriai, semantika „daugelis skaito / vienas rašo“, kai galima įtraukti asinchronines reikšmės transformacijas, net jei ji dar neprieinama. \cite{Jef15:FRPPromises}
  \item Srautai – reaktyvieji srautai: begaliniai duomenų apdorojimo srautai, įgalinantys asinchronines, neblokuojančias, kompensuojančias apkrovą transformacijos komandų grandines tarp daugybės šaltinių ir paskirties vietų.
  \item Duomenų srauto kintamieji – atskirai priskiriami kintamieji (atminties elementai), kurie gali būti priklausomi nuo įvesties, procedūrų ir kitų elementų, kad jie būtų automatiškai atnaujinami po pakeitimų. Praktinis to pavyzdys yra skaičiuoklės, kuriose pasikeitus langelio reikšmei, tai paveikia visas priklausomas funkcijas ir dėl to sukuriamos naujos reikšmės.
\end{itemize}

Populiarios bibliotekos, JVM palaikančios reaktyviojo programavimo metodus, apima, bet neapsiriboja, „Akka Streams“\footnote{http://doc.akka.io/docs/akka/2.4.3/scala/stream/index.html}, „Ratpack“\footnote{https://ratpack.io/}, „Reactor“\footnote{https://projectreactor.io/}, „RxJava“\footnote{https://github.com/ReactiveX/RxJava} ir „Vert.x“\footnote{http://vertx.io/}. Šios bibliotekos realizuoja reaktyviųjų srautų specifikaciją, kuri yra reaktyviojo programavimo bibliotekų JVM funkcinio suderinamumo standartas ir pagal aprašymą yra „...iniciatyva pateikti asinchroninio srauto apdorojimo su neblokuojančiu apkrovos kompensavimu standartą.“

\subsection{Įvykių valdomas ir pranešimų valdomas programavimas}

Kaip jau minėta, reaktyvusis programavimas (pagrįstas skaičiavimu naudojant efemeriškas duomenų srauto grandines) paprastai valdomas įvykių, o reaktyviosios sistemos (pagrįstos paskirstytųjų sistemų komunikacijos ir koordinavimo atsparumu ir lankstumu) yra valdomos pranešimų (dar vadinama duomenų siuntomis).

Pranešimų valdomos sistemos su ilgalaikiais adresuojamaisiais komponentais ir įvykių bei duomenų srautų valdomo modelio pagrindinis skirtumas yra tas, kad pranešimai yra iš principo nukreipti, o įvykiai – ne. Pranešimai turi aiškią (vieną) paskirties vietą, o įvykiai yra faktai, kuriuos kiti turi stebėti. Be to, pageidautina, kad pranešimų siuntimas būtų asinchroninis, kai siuntimas ir gavimas yra atitinkamai atskirtas nuo siuntėjo ir gavėjo.

Reaktyvumo manifestas šių sąvokų skirtumą apibrėžia taip \cite{ReactiveManifesto}:

„Pranešimas – tai duomenų elementas, siunčiamas į konkrečią paskirties vietą. Įvykis – tai signalas, komponento skleidžiamas pasiekus tam tikrą būseną. Pranešimų valdomoje sistemoje adresuojami gavėjai laukia pranešimų ir reaguoja į juos, o kitu atveju būna neveiklūs. Įvykių valdomoje sistemoje pranešimų klausytojai yra susieti su įvykių šaltiniais ir yra iškviečiami, kai paskleidžiamas įvykis. Tai reiškia, kad įvykių valdoma sistema yra pagrįsta adresuojamais įvykių šaltiniais, o pranešimų valdoma sistema koncentruojasi į adresuojamus gavėjus.“

Pranešimai reikalingi komunikacijai tinkle ir sudaro komunikacijos paskirstytosiose sistemose pagrindą, o įvykiai skleidžiami lokaliai. Pažvelgus giliau, paprastai pranešimų siuntimas naudojamas norint įveikti įvykių valdomos sistemos ribotumą tinkle, siunčiant įvykius pranešimų viduje. Tai leidžia išlaikyti santykinį įvykių valdomo programavimo modelio paskirstytajame kontekste paprastumą ir gali puikiai tikti specializuotiems ir tinkamos aprėpties naudojimo atvejams (pvz., „AWS Lambda“, paskirstytojo srauto apdorojimo produktai kaip „Spark Streaming“, „Flink“, „Kafka“ ir Akka Streams“ naudojant „Gearpump“, ir paskirstytieji publikavimo-prenumeravimo produktai kaip „Kafka“ ir „Kinesis“).

Tačiau reikalingas kompromisas: gaunamas abstrahavimas ir programavimo modelio paprastumas, tačiau prarandama kontrolė. Pranešimų siuntimas verčia mus susidurti su paskirstytųjų sistemų realybe ir apribojimais – tokiais dalykais kaip dalinės triktys, trikčių aptikimas, pamesti / besidubliuojantys / pertvarkyti pranešimai, galutinis nuoseklumas, kelių vienalaikių realybių valdymas ir kt. Pavyzdžiui, prieš pradėdamas kurti ``Akka'' produktą, Jonas Boner susidūrė su minėtomis išplečiamumo bei atsparumo problemomis\footnote{http://readwrite.com/2014/07/10/akka-jonas-boner-concurrency-distributed-computing-internet-of-things/} šiose technologijose: ``EJB'', ``RPC'', ``CORBA'' ir ``XA''.

Šie semantikos ir pritaikomumo skirtumai turi didelę įtaką programų kūrimui, įskaitant tokius dalykus kaip atsparumas, lankstumas, mobilumas, vietos skaidrumas ir kompleksiškas paskirstytųjų sistemų valdymas.

Reaktyviojoje sistemoje, ypač tokioje, kurioje naudojamas reaktyvusis programavimas, bus naudojami ir įvykiai, ir pranešimai, kadangi vienas yra puikus komunikacijos (pranešimai) įrankis, o kitas – puikus būdas išreikšti faktus (įvykiai).

\subsection{Reaktyviosios sistemos ir architektūra}

Kaip apibrėžiama Reaktyvumo manifeste, reaktyviosios sistemos – tai architektūrinių kūrimo principų rinkinys, skirtas modernioms sistemoms, galinčioms patenkinti augančius šiuolaikinių sistemų poreikius, kurti.

Reaktyviųjų sistemų principai toli gražu nėra nauji ir juos galima aptikti jau 8-ajame ir 9-ajame dešimtmetyje, reikšmingame Jim Gray darbe apie „Tandem“ sistemą \cite{Gray85whydo} ir Joe Armstrong darbuose apie „Erlang“ \cite{Armstrong:1997:DE:258948.258967}. Vis dėlto šie žmonės pralenkė laiką, kadangi tik per pastaruosius 5–10 metų technologijų pramonė buvo priversta peržiūrėti esamas įmonių sistemų kūrimo geriausias praktikas ir išmoko reaktyvumo principų žinias pritaikyti šiuolaikiniame pasaulyje, kurį apibrėžia daug branduolių, debesų kompiuterija ir internetu sąveikaujantys įrenginiai (angl. „Internet of Things“).

Reaktyviosios sistemos pagrindą sudaro pranešimų perdavimas, sukuriantis laikiną ribą tarp komponentų, leidžiančią juos atskirti laike (tai įgalina vienalaikiškumą) ir erdvėje (tai įgalina paskirstymą ir mobilumą). Toks atskyrimas būtinas norint visiškai izoliuoti komponentus ir užtikrina atsparumą bei lankstumą.

Programos aprašytos Reaktyvumo manifeste turėtų:

\begin{itemize}
  \item Reaguoti į įvykius: įvykiais paremta prigimtis įgalina sekančias sąvybes.
  \item Reaguoti į apkrovą: koncentruotis į išplečiamumą, o ne našumą vienam vartotojui.
  \item Reaguoti į trikdžius: kurti atsparias sistemas, turinčias galimybę atkurti būseną bet kuriuo metu.
  \item Reaguoti į vartotojus: apjungti paminėtus bruožus siekiant užtikrinti interaktyvią vartotojo patirtį.
\end{itemize}

% \subsection{Kaip reaktyvusis programavimas susijęs su reaktyviosiomis sistemomis?}

% Reaktyvusis programavimas yra puikus vidinės logikos ir duomenų srauto transformavimo tvarkymo metodas vietos komponentuose, kaip kodo aiškumo, našumo ir išteklių efektyvumo optimizavimo būdas. Reaktyviosios sistemos, kurios yra architektūros principų rinkinys, išryškina paskirstytuosius ryšius ir suteikia mums įrankius, skirtus spręsti paskirstytųjų sistemų tolerancijos sutrikimų ir lankstumo klausimus.

% Viena dažna tik reaktyviojo programavimo pakartotinio panaudojimo problema yra ta, kad dėl jo glaudaus sujungimo tarp skaičiavimo etapų ir įvykių valdomos, atgaliniu iškvietimu pagrįstos arba deklaratyviosios programos sunkiau užtikrinti atkuriamumo galimybę, nes transformavimo grandinės dažnai yra efemerinės ir jo etapai (atgalinio iškvietimo arba kombinatorių) yra anoniminiai, t. y. neturi adreso.

% Tai reiškia, kad paprastai sėkmė arba nesėkmė valdomos tiesiogiai, nesiunčiant signalo į išorę. Neturint adresuojamumo galimybės sudėtingiau užtikrinti atskirų etapų atkūrimą, nes paprastai neaišku, kur reikėtų arba net galima būtų paskirstyti išimtis. Todėl nesėkmės yra susijusios su efemerinėmis kliento užklausomis, o ne su bendrąja komponento sveikata – jeigu vienas iš duomenų srauto grandinės etapų nepavyksta, visą grandinę reikia paleisti iš naujo ir apie tai pranešama klientui. Priešingai nei pranešimais valdomojoje reaktyviojoje sistemoje, kuri gali pati pasitaisyti, nepranešant apie tai klientui.

% Kitas skirtumas, palyginti su reaktyviosiomis sistemomis, yra tas, kad grynas reaktyvusis programavimas leidžia atjungti laike, bet ne erdvėje (nebent pakartotinio panaudojimo pranešimų perdavimas paskirstomas duomenų srauto diagramoje tinkle, kaip aptarta anksčiau). Kaip minėta, atjungimas laike užtikrina vienalaikiškumą, atjungimas erdvėje – paskirstymą ir mobilumą (kuris leidžia ne tik statines, bet ir dinamines topologijas), kas yra ypač būtina lankstumui.

% Dėl nepakankamo vietos skaidrumo sudėtinga pritaikyti programą, remiantis tik reaktyviojo programavimo metodais taikant lanksčius modelius, todėl reikalinga papildomų įrankių, pvz., pranešimų magistralės, duomenų tinklelio arba specialių tinklo protokolų, lygmeninė sąsaja. Štai čia ir atsiskleidžia reaktyviųjų sistemų pranešimais valdomo programavimo pranašumas, nes tai yra ryšio abstrakcija, palaikanti programavimo modelį ir semantiką visuose skalės matmenyse, todėl sumažėja sistemos sudėtingumas ir pažinimo pridėtinės sąnaudos.

% Dažnai minima atgaliniu iškvietimu pagrįsto programavimo problema yra ta, kad nors rašyti tokias programas yra palyginti paprasta, tačiau ilgai naudojant atsiskleidžia tikrosios pasekmės.

% Pavyzdžiui, anoniminiais atgaliniais iškvietimais pagrįstos sistemos suteikia labai nedaug įžvalgų, kai reikia jas suprasti, tvarkyti arba, svarbiausia, išsiaiškinti, kokios, kur ir kodėl atsiranda gamybos prastovos ir netinkamas veikimas.

% Reaktyviosioms sistemoms sukurtos bibliotekos ir platformos (pvz., „Akka“ projektas ir „Erlang“ platforma) seniai išmoko šią pamoką ir remiasi ilgamečiais adresuojamaisiais komponentais, kuriuos ilgainiui paprasčiau suprasti. Įvykus trikčiai, komponentas unikaliai identifikuojamas kartu su pranešimu, kuris sukėlė triktį. Kai adresuojamumo koncepcija naudojama komponentų modelio pagrinde, stebėjimo sprendimai gali prasmingai pateikti surinktus duomenis, pakartotinai panaudojant paskirstytąsias tapatybes.

% Geros programavimo paradigmos, tokios, kuri leistų realizuoti, pvz., adresuojamumą ir trikčių tvarkymą, pasirinkimas yra neįkainojamas gamyboje, nes kuriama galvojant apie tikrą realybę, tikintis ir priimant nesėkmę, o ne bergždžiai bandant jos išvengti.

% Apskritai reaktyvusis programavimas yra labai naudingas realizavimo metodas, kurį galima naudoti reaktyviojoje architektūroje. Tačiau jis padeda tvarkyti tik dalį darbų: tvarkyti duomenų srautą naudojant asinchroninį ir neblokuojantį vykdymą (paprastai naudojant vieną mazgą arba tarnybą). Jei yra keli mazgai, reikia gerai apgalvoti tokius dalykus, kaip duomenų nuoseklumas, ryšys tarp mazgų, koordinavimas, versijos kūrimas, tvarkymas, trikčių tvarkymas, interesų ir atsakomybių atskyrimas ir pan., t. y. apgalvoti sistemos architektūrą.

% Todėl norėdami maksimaliai padidinti reaktyviojo programavimo reikšmę, jis naudojamas kaip vienas iš įrankių reaktyviajai sistemai kurti. Kuriant reaktyviąją sistemą reikia ne tik išgauti OS būdingus išteklius, šiek tiek asinchroninių API ir grandinių pertraukiklių ant esamos, senstelėjusios programinės įrangos paketo viršaus. Reikia nuolat prisiminti faktą, kad kuriate paskirstytąją sistemą, kurią sudaro kelios tarnybos, turinčios veikti kartu, užtikrinant nuoseklią ir atsakomąją patirtį, ne tik tuomet, kai viskas veikia taip, kaip numatyta, bet ir įvykus trikčiai ir esant nenuspėjamai apkrovai.

\subsection{Reaktyviojo programavimo pranašumai}

Pagrindiniai reaktyviojo programavimo pranašumai: didesnis skaičiavimo išteklių naudojimas kelių branduolių ir kelių CPU aparatūroje ir didesnis našumas sumažinant nuoseklinimo taškus pagal Amdalo dėsnį \cite{Rodgers:1985:IMS:327070.327215} ir, žiūrint plačiau, pagal Giunterio universaliojo išplečiamumo dėsnį \cite{opac-b1118020}.

Papildomas pranašumas yra programų kūrėjo produktyvumas, kadangi tradicinėms paradigmoms sunkiai sekėsi pateikti paprastą ir įgyvendinamą metodą asinchroniniam ir neblokuojančiam skaičiavimui ir I/O apdoroti. Reaktyvusis programavimas susitvarko su daugeliu iš šių iššūkių, kadangi jis paprastai pašalina tiesioginio koordinavimo tarp aktyvių komponentų poreikį.

Reaktyvusis programavimas ypač sėkmingai taikomas komponentų kūrimo ir darbo eigų komponavimo srityse. Norint išnaudoti visas asinchroninio vykdymo galimybes ir išvengti perteklinio naudojimo arba begalinio išteklių naudojimo, būtina įtraukti apkrovos kompensavimą.

Nors reaktyvusis programavimas yra labai naudingas kuriant modernią programinę įrangą, galvojant apie sistemą aukštesniu lygmeniu, reikia naudoti kitą įrankį – reaktyviąją architektūrą – reaktyviųjų sistemų kūrimo procesą. Taip pat svarbu prisiminti, kad yra daugybė programavimo paradigmų ir reaktyvusis programavimas yra tik viena iš jų. Kaip ir bet koks kitas įrankis, jis nėra skirtas visiems galimiems atvejams.

\subsection{Reaktyviojo programavimo svarba šiais laikais}

Paskelbus Reaktyvumo manifestą, „Scala“ kūrėjas Martin Odersky, reaktyviųjų plėtinių kūrėjas Erik Meijer ir „Akka“ techninis vadovas Roland Kuhn „Coursera“ paskelbė nemokamą kursą „Principles of Reactive Programming“\footnote{https://www.lightbend.com/blog/principle-of-reactive-programming-coursera} (Reaktyviojo programavimo principai):

„Antrojo kurso tikslas – išmokti reaktyviojo programavimo principus. Reaktyvusis programavimas yra naujai atsirandantis mokymo dalykas, apimantis vienalaikiškumą ir įvykiais pagrįstas bei asinchronines sistemas. Jis būtinas rašant bet kokio tipo žiniatinklio paslaugą arba paskirstytąją sistemą ir taip pat sudaro daugelio didelio našumo vienalaikių sistemų pagrindą. Reaktyvųjį programavimą galima vertinti kaip aukštesnio lygmens funkcinio programavimo vienalaikių sistemų natūralų plėtinį, kuris tvarko paskirstytąją būseną, koordinuodamas ir organizuodamas asinchroninių duomenų srautus, kuriais keičiasi veikėjai.“

RP plačiai naudoja „Netflix“ \cite{Netflix}, ji netgi skyrė „Rx“ „Java“ prievadą:

„Reaktyvusis programavimas su „RxJava“ leido „Netflix“ kūrėjams panaudoti serverio vienalaikiškumą, nepatiriant įprastų su gijomis susijusių ir sinchronizavimo problemų. API paslaugos lygmens realizavimas kontroliuoja vienalaikiškumo primityvus, kurie leidžia mums gerinti sistemos našumą, nesibaiminant sugadinti kliento kodo. „RxJava“ efektyviai veikia mums serveryje ir kuo daugiau naudojame, tuo giliau įsiskverbia į mūsų kodą.“

2013 metais „Facebook“ taip pat išleido „React JavaScript“ biblioteką, skirtą kitos kartos vartotojo sąsajoms kurti. „Facebook“ inžinierius Stoyan Stefanov aprašo pagrindinę „React“ koncepciją \cite{React}:

„React“ leidžia kurti programą naudojant komponentus, kurie gali atvaizduoti kai kuriuos duomenis. Kai duomenys pasikeičia, komponentai labai efektyviai ir tik tuomet, kai reikia, automatiškai atnaujinami. O visomis įvykių doroklių prijungimo ir atjungimo užduotimis pasirūpinama už jus. Taip pat efektyvu naudoti ir perdavimą.“

Verta pastebėti, kad RP jau plačiai naudoja vartojo sąsajos kūrėjų bendruomenė, kurie maždaug 2009 m. pradėjo naudoti originalų „Flapjax“ dokumentą \cite{Meyerovich:2009:FPL:1639949.1640091} ir toliau naudojo kelias bibliotekas, kuriose buvo realizuoti RP principai, pvz., „Bacon.js“\footnote{https://baconjs.github.io/}, „Knockout“\footnote{http://knockoutjs.com/}, „Meteor“\footnote{https://www.meteor.com/}, „React.js“\footnote{https://facebook.github.io/react/}, „Reactive.coffee“\footnote{http://yang.github.io/reactive-coffee/} ir „RxJS“\footnote{http://reactivex.io/}.
