\section{``Frappuccino'' bibliotekos išeities kodo pavyzdžiai} \label{appendix_frappuccino}

\begin{lstlisting}[caption=``Frappuccino'' bibliotekos panaudojimo pavyzdys, label=frappuccino]
class Button
  def push
    emit(:pushed) # emit sends a value into the stream
  end
end

button = Button.new
stream = Frappuccino::Stream.new(button)

counter = stream
            .map { |event| event == :pushed ? 1 : 0 } # convert events to ints
            .inject(0) { |sum, n| sum + n } # add them up

counter.now # => 0

button.push
button.push
button.push

counter.now # => 3

button.push

counter.now # => 4
\end{lstlisting}

\section{Projekto, panaudojančio``RailsEventStore'' biblioteką išeities kodo pavyzdžiai} \label{appendix_rails_event_store}

\begin{lstlisting}[caption=Įvykių saugojimo migracija, label=event_store_events_table]
    create_table(:event_store_events) do |t|
      t.string      :stream,      null: false
      t.string      :event_type,  null: false
      t.string      :event_id,    null: false
      t.text        :metadata
      t.text        :data,        null: false
      t.datetime    :created_at,  null: false
    end
    add_index :event_store_events, :stream
    add_index :event_store_events, :created_at
    add_index :event_store_events, :event_type
    add_index :event_store_events, :event_id, unique: true
\end{lstlisting}

\begin{lstlisting}[caption=``EventStore'' bibliotekos prenumeratorių mechanizmo panaudojimas, label=event_store_subscription]
class CartReadModel
    def call(event)
        if event.event_type == 'AddToCart'
            add_to_cart(event.data)
        end
        if event.event_type == 'RemoveFromCart'
            remove_from_cart(event.data)
        end
    end
    private
    def add_to_cart(event_data)
        #Implementation here
    end
    def remove_from_cart(event_data)
        #Implementation here
    end

end


cart = CartReadModel.new
client.subscribe(cart, [AddToCart, RemoveFromCart])

\end{lstlisting}

\begin{lstlisting}[caption=Įvykių pritaikymas domeno objektui, label=apply_domain_events]
module Domain
  class Order
    include AggregateRoot::Base

    AlreadyCreated    = Class.new(StandardError)
    MissingCustomer   = Class.new(StandardError)

    def initialize(id = SecureRandom.uuid)
      @id = id
      @state = :draft
    end

    def create(order_number, customer_id)
      raise AlreadyCreated unless state == :draft
      raise MissingCustomer unless customer_id
      apply Events::OrderCreated.create(@id, order_number, customer_id)
    end

    # ...

    def apply_order_created(event)
      @customer_id = event.customer_id
      @number = event.order_number
      @state = :created
    end

    private
    attr_accessor :id, :customer_id, :order_number, :state

    # ...
  end
end
\end{lstlisting}

\begin{lstlisting}[caption=``AggregateRoot'' modulis, label=aggregate_root]
module AggregateRoot
  class DefaultApplyStrategy
    def call(aggregate, event)
      event_name_processed = event.class.name.demodulize.underscore
      aggregate.method("apply_#{event_name_processed}").call(event)
    end
  end
end

module AggregateRoot
  def apply(event)
    apply_strategy.(self, event)
    unpublished_events << event
  end

  private

  def unpublished_events
    @unpublished_events ||= []
  end

  def apply_strategy
    DefaultApplyStrategy.new
  end
end
\end{lstlisting}
