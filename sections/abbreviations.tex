\begin{itemize}

  \item \textbf{API} (angl. Application programming interface) - rinkinys funkcijų ar procedūrų, leidžiančių kurti programas, kurios turi prieigą prie kitos operacinės sistemos, programos ar paslaugos informacijos ar tam tikro funkcionalumo.

  \item \textbf{ES} (angl. Event Sourcing) - įvykių kaupimo principas.

  \item \textbf{ES+CQRS} - įvykių kaupimo bei komandų-užklausų atskyrimo principai, kurie dažniausiai yra naudojami kartu projektuojant sistemą.

  \item \textbf{CQS} (angl. Command Query Separation) - komandų-užklausų atskyrimas.

  \item \textbf{CQRS} (angl. Command Query Responsibility Segregation) – komandų-užklausų atsakomybių atskyrimas.

  \item \textbf{DDD} (angl. Domain-Driven Design) – būdas kurti programinę įrangą, skirtą spręsti sudėtingus uždavinius, bei apjungti realizaciją kartu su augančiu domeno modeliu.

\end{itemize}
