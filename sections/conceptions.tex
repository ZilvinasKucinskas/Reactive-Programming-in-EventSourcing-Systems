\begin{itemize}

  \item \textbf{Agregatas} (angl. aggregate) - DDD modelis, rinkinys domeno objektų, kurie gali būti laikomi kaip visuma.

  \item \textbf{Nuoseklumas arba darna} (angl. consistency) - Duomenų ir jų savybių, esančių toje pačioje sistemoje, logiškumas, vieno su kitu suderintumas.

  \item \textbf{Derinimas} (angl. debugging) - riktų bei klaidų paieška programinėje įrangoje bei jų taisymas.

  \item \textbf{Išplečiamumas} (angl. scaling) – galimybė sujungti daugybę techninės ar programinės įrangos esybių taip, jog jos dirbtų kaip visuma. Pavyzdžiui, galima pridėti keletą serverių pasinaudojant grupavimu arba apkrovos paskirstymu taip pagerinant sistemos našumą bei prieinamumą.

  \item \textbf{Introspekcija} (angl. introspection) - programos gebėjimas nagrinėti objekto tipą ir sąvybes programos vykdymo metu.

  \item \textbf{Metodų jungimo principas} (angl. method chaining) - objektinės programavimo kalbos sintaksė, leidžianti iškviesti kelis metodus iš eilės. Kiekvienas metodas grąžina objektą, taip leidžiant jungti metodų kvietimus viename apraše ir nereikalaujant papildomų kintamųjų tarpinės būsenos saugojimui.

  \item \textbf{Modulis} (angl. module) - savarankiška programos dalis, į kurią sudėti duomenys ir veiksmai su jais.

  \item \textbf{Refleksija} (angl. reflection) - programos gebėjimas tiek nagrinėti objektus (introspekcija), tiek modifikuoti save (elgesį arba būseną) programos vykdymo metu.

\end{itemize}
