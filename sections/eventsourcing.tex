Šis skyrius aprašo įvykių kaupimą, susijusią terminologiją. privalumus bei trūkumus, įvykių srautus bei panaudojimo atvejus. Taip pat aprašomas architektūrinis stilius - komandų-užklausų atsakomybių atskyrimas (CQRS), kuris yra naudojamas įvykių kaupimu pagrįstose sistemose.


\subsection{Aktyvus įrašas (angl. Active Record)}

Dauguma programų sistemų reikalauja tam tikro duomenų valdymo. Įprastas būdas išlaikyti dabartinę programos būseną yra naudojant tam tikrą duomenų saugyklą (pvz.: duomenų bazę) ir gražinti saugomų objektų būseną panaudojant užklausą. Aktyvus įrašas yra programinės įrangos kūrimo šablonas darbui su dabartine programos būsena pritaikant sukūrimo, skaitymo, atnaujinimo ir ištrynimo operacijas (CRUD) objektams reliacinėje duomenų bazėje \cite{Fowler:2002:PEA:579257} (Paveikslėlis \ref{img:current_state}.a). Taikant šį programinės įrangos kūrimo šabloną, tik dabartinė objekto būsena yra palaikoma ir manipuliuojama, buvusios objekto reikšmės yra perrašomas ir neišsaugoti pakeitimai prarandami. Šis kelias puikiai tinka daugumai programų, bet turi trūkumų kalbant apie veiksmų atsekamumą bei operacijas darbui su programos būsenų istorija.

\subsection{Įvykių kaupimo apibrėžimas}

Akademinė literatūra šia tema yra ganėtinai reta. Daugiausia informacijos galima rasti internetiniuose dienoraščiuose, prezentacijose bei programinės įrangos dokumentacijose. Šia tema nėra standartizuoto žodyno, terminologija ir apibrėžimai skiriasi priklausomai nuo autoriaus.

Pirmą kartą įvykių kaupimo terminas 2005 metais paminėjo Martin Fowler savo internetiniame dienoraštyje \cite{Fowler:EventSourcing}. Jis aprašo įvykius kaip eilę programos būsenų pasikeitimų. Šie įvykiai saugo visą informaciją, reikalingą dabartinės būsenos atkūrimui. Įvykiai niekada neištrinami. Vienintelis būdas anuliuoti įvykį yra sukurti grįžtamąjį įvykį (angl. retroactive event) \cite{Fowler:RetroactiveEvent}. Grįžtamasis įvykis grąžina programos būseną į tokia būseną lyg praeitas įvykis nebūtų nutikęs.

Kitas žymus autorius minint įvykių kaupimą yra Greg Young. Jis apibūdina įvykių kaupimą kaip “dabartinės būsenos saugojimą kaip eilę įvykių bei sistemos būsenos atkūrimą pakartojant tą pačią įvykių seką“ \cite{Young:CQRS2010}. Jo požiūriu, įvykių žurnalas skirtas tik įrašymui: “įvykiai yra lyg faktai. Jie įvyko ir negali būti anuliuoti” \cite{Young:CQRS2013}. Tai ką Martin Fowler vadina grįžtamaisiais įvykiais, Greg Young apibūdina kaip grįžtamąsias operacijas (angl. reversal transactions).

Martin Krasser straipsniai bei prezentacijos apie Akka priemonių rinkinį įrašymo moduliui aprašo dar vieną požiūrį į įvykių kaupimą \cite{Krasser:AkkaPersistence, Krasser:AkkaYoutube}. Šiame kontekste, išsiskaidžiusioje sistemoje aktoriai komunikuoja žinutėmis, kurios pakeičia būseną. Įvykių kaupimas naudojamas perduoti pakitimus aktoriui. Būsenos pakitimai yra pridedami kaip nesikeičiantys faktai į įvykių žurnalą. Šis sprendimas yra motyvuojamas tuo, jog “šis būdas leidžia pasiekti labai aukštą operacijų kiekį ir įgalina efektyvias replikacijas”. Aktoriaus būsenos atkūrimas (po perkrovimo ar klaidos) yra pasiekiamas pritaikant jau įrašytus įvykius.

Visiems apibrėžimams yra būdingas vienas bruožas - publikuoti kiekvieną objekto (sistemos ar programos) būsenos pasikeitimą kaip nesikeičiantį įvykį į nemodifikuojamą žurnalą. To pasekoje skaitant įvykius iš eilės galima atkurti dabartinę būseną (Paveikslėlis \ref{img:current_state}.b).

\begin{figure}[H]
    \centering
    \includegraphics[scale=0.8]{img/current_state}
    \caption{Du skirtingi būdai saugoti dabartinę būseną}
    \label{img:current_state}
\end{figure}

\subsection{Įvykių kaupimo privalumai}

Literatūros šaltiniuose dažnai randama rekomendacijų taikyti įvykių kaupimą tik tam tikrose, aiškiai apibrėžtuose sistemos dalyse ir nenaudoti šio principo ten, kur tai nėra svarbu \cite{Betts:2013:ECE:2509680}. Sudėtinga verslo logika yra dažnas įvykių kaupimu pagrįstų sistemų bruožas. Tai yra priešingybė pavyzdžiui programoms, kurios suteikia galimybę keisti reikšmes vartotojo sąsajoje ir saugoti reliacinėje duomenų bazėje. Kad būtų galima įvertinti taikymo galimybes, reikia aiškiai apibrėžti, kokius privalumus gali suteikti įvykiu kaupimo principu pagrįstos architektūros.

\begin{enumerate}

  \item \textbf{Audito žurnalas}. Reguliuojamose srityse (pvz. finansų industrija), daugelyje šalių valstybės nuostatai reikalauja kompanijų saugoti operacijų istoriją sistemoje. Pavyzdžiui, JAV reikalauja tarpininkų saugoti įrašus neperrašomu ir neištrinamu formatu \cite{US:StorageRules}. Įvykių kaupimo principas puikiai tinka šiam reikalavimui įgyvendinti, nes įvykių žurnalas pasižymi tik įrašymo funkcionalumu, o patys įvykiai yra nekintantys.

  \item \textbf{Derinimas}. Sukaupti įvykiai gali būti panaudoti analizei kaip sistema pasiekė vieną ar kitą būseną ir kurie įvykiai tai įtakojo. Įvykių kaupimo stiprioji pusė yra atsekamumas ir derinimo galymybės - įmanoma atsekti iš kur kilo sistemos klaida.

  \item \textbf{Išplečiamumas}. Faktas, jog į įvykių žurnalą galima tik įrašyti, yra naudingas išplečiamoms architektūroms. Gana dažnai įvykių kaupimo principu paremtose sistemose turima keletą duomenų modelio kopijų. Norint užtikrinti darną šios kopijos turi būti sinchronizuotos. Tokiose sistemose vienintelis būdas suvienodinti duomenų modelio kopijas yra įrašant įvykius. Manoma, jog toks sprendimas turi mažiau blokavimų ir lengviau išplečiamas skaitymui negu architektūra atnaujinant duomenų modelį \cite{GetEventStore:Basics}.

  \item \textbf{Informacinė nauda}.
 Visos buvusios sistemos būsenos gali būti tiek atkurtos, tiek gaunamos užklausų pagalba. Tai gali atnešti papildomos naudos sistemoms, kuriose kliento elgesys yra svarbus. Tokiose sistemose dažniausiai nežinoma kokio tyrimo reikės iš verslo pusės. Pavyzdžiui, elektroninės komercijos parduotuvėje gali būti naudinga gauti ir nagrinėti visus produktus, kurie kada nors buvo išimti iš apsipirkimo krepšelio. Įvykių kaupimo sistemose, toks uždavinys gali būti lengvai įgyvendintas.

\end{enumerate}

\subsection{Panaudojimo atvejai}

Internete galima rasti straipsnių aprašančių kaip įvykių kaupimo principas buvo panaudotas tikrose sistemose. Galima pažymėti dvi išsamias dokumentacijas: finansinės prekybos platforma LMAX \cite{Fowler:LMAX} ir Microsoft Windows Azure komandos analizė \cite{Betts:2013:ECE:2509680}. Taipogi yra sukurti keli pragramavimo karkasai paremti įvykių kaupimu: Akka\footnote{http://akka.io/} ir Event Store\footnote{http://docs.geteventstore.com/} bei Red Bull Media House kompanijos programavimo priemonių rinkinys Eventuate\footnote{https://rbmhtechnology.github.io/eventuate/}.

\subsection{Komandų-užklausų atskyrimo principas (angl. Command Query Separation (CQS))}

Komandų-užklausų atskyrimo principą pirmą kartą aprašė Bertrand Meyer \cite{Meyer:1988:OSC:534929} norėdamas patobulinti šalutinių efektų apdorojimą kuriant programą ar projektuojant API. Pagrindinė idėja yra atskirti prieigą prie objektų į:

\begin{enumerate}

  \item \textbf{Užklausas}, kurios grąžina informaciją,

  \item \textbf{Komandas}, kurios keičia būseną (Paveikslėlis \ref{img:cqs}.a).

\end{enumerate}

Užklausos neturėtų sukelti pašalinių efektų. Kitais žodžiais tariant - klausimas neturėtų pakeisti atsakymo.

\begin{figure}[H]
    \centering
    \includegraphics[scale=0.6]{img/cqs}
    \caption{CQS ir CQRS palyginimas}
    \label{img:cqs}
\end{figure}

\subsection{Komandų-užklausų atsakomybių atskyrimas (angl. Command Query Responsibility Segregation (CQRS))}

Komandų-užklausų atsakomybės atskyrimas – šablonas, kurį pirmasis aprašė Greg Young \cite{Young:CQRS2013}. Kai kuriuose ankstesniuose šaltiniuose šis šablonas aprašomas kaip plėtinys arba specialus komandų-užklausų atskyrimo principo atvejis, tačiau šiandien jis laikomas atskiru šablonu, sukurtu CQS principu. Pagal CQRS teigiama, kad užklausoms vykdyti naudojamas skirtingas modelis (tai yra komponentas arba objektas) nei modelis, skirtas komandoms vykdyti (Paveikslėlis \ref{img:cqs}.b). CQS padalija atsakomybę pagal kodo lygį, suskirstydamas metodus į užklausas ir komandas; CQRS analizuoja tai dar išsamiau ir netgi skirsto objektus į du tipus: skaitymo arba rašymo. G. Young aprašė šabloną taip: „CQRS yra paprastas dviejų objektų sukūrimas iš anksčiau buvusio vieno objekto“ \cite{Young:CQRS2010}.

Kuriant architektūrą pagal CQRS šabloną, ilgainiui sistemoje sukuriamas nuoseklus veikimas. Užklausos ir komandos modelis yra atskiri komponentai, todėl, atsižvelgiant į jų sujungimą, jie nebūtinai turi būti sinchroniški. Jei jie sujungiami laisvai, užklausos modelyje nebūtinai kaskart bus tie patys duomenys kaip komandos modelyje. Ši charakteristika, kai užklausos modelyje gali būti laikinai nenuosekli (tai yra pasenusi) būsena, kuri kažkuriame ateities taške ilgainiui taps nuosekli, yra vadinama galutiniu nuoseklumu. Šis galutinai nuoseklus veikimas tampa itin svarbus paskirstytosioms sistemoms, kai užklausos modelis (-iai) yra fiziškai atskiriamas (-i) nuo komandos modelio (-ių). Brewer CAP teorema \cite{Fox:1999:HYS:822076.822436} teigia, kad paskirstytoji sistema turi neišvengiamai nuspręsti, kaip apdoroti nuoseklumą, nes, įvykus tinklo klaidai, ji gali užtikrinti tik dvi iš šių trijų garantijų:

\begin{itemize}
  \item Nuoseklumą: visi dalyviai peržiūri tuos pačius duomenis.
  \item Pasiekiamumą: atsakoma į kiekvieną užklausą.
  \item Atskyrimo toleravimą: sistema ir toliau veikia net tada, jei tarp sistemos savavališkų dalyvių prarandami pranešimai. Tai turi poveikį nuoseklumui, nes dalyviai turi sutvarkyti pranešimų praradimą. Be to, tai veikia pasiekiamumą, nes kiekvienas dalyvis turi vykdyti užklausas.
\end{itemize}

Taigi, jei paskirstytoji sistema kuriama kaip architektūros stilių naudojant CQRS, o vietoj nuoseklumo pasirenkamas atskyrimo toleravimas ir pasiekiamumas, sistema gali generuoti tik galutinį nuoseklumą. Nuslėpus faktą, kad sistema veikia galutinai nuosekliai (tai yra numatant optimistines prielaidas), gali būti daroma pavojinga klaida, nes programuotojai ir programinės įrangos komponentai tuomet veiktų pagal klaidingas prielaidas. Progresyvus nuoseklumo problemų sprendimas gali būti laikomas paskirstytosios ES+CQRS sistemos stipriąja puse, nes jis verčia programą kuriančius programuotojus atsižvelgti į tokį nuoseklų veikimą, kuris yra integruota bendros architektūros dalis. Užklausos gali visą laiką pateikti pasenusį rezultatą ir nebeaišku, kada vykdomos komandos. Pastaraisiais metais šis progresyvus požiūris į didelių paskirstytųjų sistemų problemas sukūrė tam tikrą požiūrio kampą ir sugeneravo kelis projektus. Reaktyvumo manifestas \cite{ReactiveManifesto} yra svarbus 2013 m. pasirodęs dokumentas. Jame apžvelgiamos pagrindinės ypatybės, kuriomis pasižymi paskirstytosios sistemos projektavimas, atitinkantis įvykių teikimą ir CQRS.

Komandos ir užklausos atskyrimas kelia vien nepatogumą, nes jis lemia sudėtingesnį sistemos kūrimą ir tampa sunku sukurti priežastiniu būdu susijusias komandų ir įvykių serijas. Komandos veikia asinchroniškai ir nepateikia reikšmės, todėl programa neturi priemonių sužinoti, kada matomas komandos rezultatas. Taigi, po komandos einanti užklausa gali pateikti pasenusį rezultatą. Tačiau paskesnės užklausos galiausiai kažkuriuo metu pateiks nuoseklią būseną. 2.4 pav. pateikiama tipiška komandų ir užklausų seka. Tokį veikimą galima apeiti, pavyzdžiui, pavėlinus užklausų vykdymą, kol užklausos modelis bus atnaujintas į tam tikrą versiją (kaip panaudota sąlyginėse užklausose dalyje „Eventuate“). Tačiau reiktų pastebėti, kad šiuos apėjimo metodus galima naudoti tik retais konkrečiais atvejais, nes naudojant šiuos metodus sumažėja efektyvumas ir reikia daugiau dirbti su šių sistemų charakteristikomis. Geriau būtų, jei programa natūraliai galėtų apdoroti galutinai nuoseklų veikimą.

\subsection{Įvykių kaupimo principas ir CQRS}

CQRS – šablonas, suformuojantis simbiotinį ryšį su įvykių kaupimu ir dažnai yra naudojamas įvykių kaupimu pagrįstoje architektūroje. Operacijų klasifikacija būseną keičiančiose komandose ir tik skaityti skirtose užklausose puikiai dera su koncepcijomis, randamomis įvykių kaupimu pagrįstose sistemose. CQRS šablone komandos nurodo tikslus, kurie siunčiami į komandų rengyklę. Čia jos apdorojamos ir pateikiami būsenos pakeitimai – įvykiai. Tada įvykiai įtraukiami į įvykių žurnalą ir publikuojami užklausų modeliuose, kur jų būsena taip pat atnaujinama. Užklausų modeliai vykdo tik skaityti skirtas užklausas ir gali pateikti tam tikrą duomenų rodinį (pvz., sąskaitų lentelę ir jų likučius). Įvykiai naudojami kaip modelių sinchronizavimo priemonė. Taigi komandos aiškiai atskiriamos nuo kitų operacijų, kurios nepateikia papildomos operacijos, jos pridedamos įvykių žurnale. 2.5 pav. pateikiama tipiška ES+CQRS architektūra: verslo logika glūdi aukštesniame taikomosios programos lygmenyje; komandos ir užklausos slepia diegimo informaciją apie tai, kaip jos vykdomos. 2.6 pav. pateikiamas konceptualus rodinys, kuriame dėmesys skiriamas vidiniams darbams.

Komandų-užklausų atskyrimas sugeneruoja charakteristikas, kurios tinkamai veikia su įvykių kaupimu. Įvykiai nekeičiami ir pridedami tik po to, kai įvykdoma komanda, todėl įvykiai yra priemonė, leidžianti išlaikyti duomenų modelių sinchronizavimą. Užklausų modeliams tereikia gauti naujų įvykių, kad jie išliktų suderinti su komandos modeliu. Padalijus problemas objekto lygiu, palaikomas kiekvieno modelio atskiras optimizavimas. Tai itin svarbi savybė tų sistemų, kurioms taikomas reikalavimas keisti dydį, nes skaitymo ir rašymo optimizavimo technikos skiriasi. Atskyrus komandos ir užklausos modelius, tampa įmanoma jų dydį keisti atskirai. Vienas iš būdų optimizuoti skaitymo operacijas priskirtojoje sistemoje – duomenų bazės dubliavimas. Tada šie duomenų bazės dublikatai turi būti sinchronizuojami, tačiau kiekvienas iš jų gali siųsti atsakymus į užklausas. Antra vertus, optimizuoti rašymą reiškia sumažinti perteklių (ir rašymą) duomenų bazėje. To galima pasiekti atitinkamai sutvarkius vieną duomenų bazę naudojant, pvz., normalizavimo taisykles.

\subsection{DDD}

Tiek įvykių kaupimas, tiek CQRS šablonas yra glaudžiai susiję su domenu pagrįstu projektavimu. DDD terminus ir koncepcijas 2003 m. aprašė Eric Evans \cite{evans2004domaindriven}. Iš esmės DDD suteikia mokomųjų principų ir komponentų rinkinį, skirtą programinės įrangos kūrimui. Pagrindinė idėja – skirti daugiausia dėmesio programinės įrangos projektui domene ir jame atliekamoms užduotims. Pagrindinė sistemos projektavimo dalis yra domeno modelio sukūrimas. Be kitų komponentų, šis modelis aprašo pagrindinio domeno užduotis ir įvykius. Šis domeno modelis sukurtas bendradarbiaujant su domenų specialistais. Todėl DDD yra panašus į aktyvius programavimo modelius, kuriuose taip pat skatinamas bendradarbiavimas su domenų specialistais (arba klientais). DDD būtina apibrėžti bendrąja kalba (skvarbiąja kalba), kuri naudojama projektavimo metu. Manoma, kad ši svarbioji kalba sukuria bendrąjį rodinį komandos nariams domeno modelyje. Veiksmažodžiai ir daiktavardžiai padeda aiškiai suprasti užduotis, kurias turi atlikti sistema. Be to, manoma, kad bendroji kalba apsaugo nuo nesusipratimų ir leidžia kiekvienam projekto nariui suprantamai kalbėtis su kitu nariu taip suvienydama domeno specialistus ir programuotojus bendram darbui.

DDD taikymas tik šiam susietam kontekstui ir aukšto lygio koncepcijų naudojimas siekiant sukurti abstrakciją iš žemo lygmens informacijos, sukuria sąsajas su įvykių kaupimo sritimi. Įvykių kaupimu pagrįstoje sistemoje paprastai nėra saugomi visi žemo lygio vykstantys įvykiai, čia labiau laikomi tik gana susiję riboto konteksto domeno įvykiai. Manoma, DDD tiktų sistemoms su sudėtingomis verslo taisyklėmis ir aiškiai apibrėžtu bei apribotu domenu. Tačiau paprasta sistema dėl DDD gali tapti sudėtinga ir mažinanti efektyvumą.

Įvykių kaupimo ir CQRS koncepcijos sukurtos iš domeno pagrįsto projektavimo. ES+CQRS sistemoje esančios komandos ir užklausos gali būti modeliuojamos naudojant DDD. Taip jos įgyja aiškią reikšmę domeno modelyje ir yra žinomos programuotojams ir domenų specialistams. Taip pat yra ir su sukurtais įvykiais – kaip domeno įvykiai jie tinka domeno modeliui ir jo logikai. Domeno specialistas aprašo komandas ir užklausas bei jų vidinę logiką. Tada programos kūrėjas per API gali pasiekti šias komandas ir užklausas ir įgyvendinti jas programoje. Komandų ir užklausų abstrakcijos kuriamos iš domenui būdingų sąlygų, todėl darbas su sistema palengvėja. Programos kūrėjas neturi atsižvelgti į tai, kaip komanda tikrinama, arba žinoti, kaip ji įgyvendinama viduje. Sudėtinga verslo logika gaunama pasitelkus supaprastintų komandų naudojimą.
