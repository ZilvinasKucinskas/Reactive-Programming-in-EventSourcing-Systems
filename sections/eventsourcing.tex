Šis skyrius aprašo įvykių kaupimą, susijusią terminologiją. privalumus bei trūkumus, įvykių srautus bei panaudojimo atvejus. Taip pat aprašomas architektūrinis stilius - komandų-užklausų atsakomybių atskyrimas (CQRS), kuris yra naudojamas įvykių kaupimu pagrįstose sistemose.


\subsection{Aktyvus įrašas (angl. Active Record)}

Dauguma programų sistemų reikalauja tam tikro duomenų valdymo. Įprastas būdas išlaikyti dabartinę programos būseną yra naudojant tam tikrą duomenų saugyklą (pvz.: duomenų bazę) ir gražinti saugomų objektų būseną panaudojant užklausą. Aktyvus įrašas yra programinės įrangos kūrimo šablonas darbui su dabartine programos būsena pritaikant sukūrimo, skaitymo, atnaujinimo ir ištrynimo operacijas (CRUD) objektams reliacinėje duomenų bazėje \cite{Fowler:2002:PEA:579257} (Paveikslėlis \ref{img:current_state}.a). Taikant šį programinės įrangos kūrimo šabloną, tik dabartinė objekto būsena yra palaikoma ir manipuliuojama, buvusios objekto reikšmės yra perrašomas ir neišsaugoti pakeitimai prarandami. Šis kelias puikiai tinka daugumai programų, bet turi trūkumų kalbant apie veiksmų atsekamumą bei operacijas darbui su programos būsenų istorija.

\subsection{Įvykių kaupimo apibrėžimas}

Akademinė literatūra šia tema yra ganėtinai reta. Daugiausia informacijos galima rasti internetiniuose dienoraščiuose, prezentacijose bei programinės įrangos dokumentacijose. Šia tema nėra standartizuoto žodyno, terminologija ir apibrėžimai skiriasi priklausomai nuo autoriaus.

Pirmą kartą įvykių kaupimo terminas 2005 metais paminėjo Martin Fowler savo internetiniame dienoraštyje \cite{Fowler:EventSourcing}. Jis aprašo įvykius kaip eilę programos būsenų pasikeitimų. Šie įvykiai saugo visą informaciją, reikalingą dabartinės būsenos atkūrimui. Įvykiai niekada neištrinami. Vienintelis būdas anuliuoti įvykį yra sukurti grįžtamąjį įvykį (angl. retroactive event) \cite{Fowler:RetroactiveEvent}. Grįžtamasis įvykis grąžina programos būseną į tokia būseną lyg praeitas įvykis nebūtų nutikęs.

Kitas žymus autorius minint įvykių kaupimą yra Greg Young. Jis apibūdina įvykių kaupimą kaip “dabartinės būsenos saugojimą kaip eilę įvykių bei sistemos būsenos atkūrimą pakartojant tą pačią įvykių seką“ \cite{Young:CQRS2010}. Jo požiūriu, įvykių žurnalas skirtas tik įrašymui: “įvykiai yra lyg faktai. Jie įvyko ir negali būti anuliuoti” \cite{Young:CQRS2013}. Tai ką Martin Fowler vadina grįžtamaisiais įvykiais, Greg Young apibūdina kaip grįžtamąsias operacijas (angl. reversal transactions).

Martin Krasser straipsniai bei prezentacijos apie Akka priemonių rinkinį įrašymo moduliui aprašo dar vieną požiūrį į įvykių kaupimą \cite{Krasser:AkkaPersistence, Krasser:AkkaYoutube}. Šiame kontekste, išsiskaidžiusioje sistemoje aktoriai komunikuoja žinutėmis, kurios pakeičia būseną. Įvykių kaupimas naudojamas perduoti pakitimus aktoriui. Būsenos pakitimai yra pridedami kaip nesikeičiantys faktai į įvykių žurnalą. Šis sprendimas yra motyvuojamas tuo, jog “šis būdas leidžia pasiekti labai aukštą operacijų kiekį ir įgalina efektyvias replikacijas”. Aktoriaus būsenos atkūrimas (po perkrovimo ar klaidos) yra pasiekiamas pritaikant jau įrašytus įvykius.

Visiems apibrėžimams yra būdingas vienas bruožas - publikuoti kiekvieną objekto (sistemos ar programos) būsenos pasikeitimą kaip nesikeičiantį įvykį į nemodifikuojamą žurnalą. To pasekoje skaitant įvykius iš eilės galima atkurti dabartinę būseną (Paveikslėlis \ref{img:current_state}.b).

\begin{figure}[H]
    \centering
    \includegraphics[scale=0.8]{img/current_state}
    \caption{Du skirtingi būdai saugoti dabartinę būseną}
    \label{img:current_state}
\end{figure}

\subsection{Įvykių kaupimo privalumai}

Literatūros šaltiniuose dažnai randama rekomendacijų taikyti įvykių kaupimą tik tam tikrose, aiškiai apibrėžtuose sistemos dalyse ir nenaudoti šio principo ten, kur tai nėra svarbu \cite{Betts:2013:ECE:2509680}. Sudėtinga verslo logika yra dažnas įvykių kaupimu pagrįstų sistemų bruožas. Tai yra priešingybė pavyzdžiui programoms, kurios suteikia galimybę keisti reikšmes vartotojo sąsajoje ir saugoti reliacinėje duomenų bazėje. Kad būtų galima įvertinti taikymo galimybes, reikia aiškiai apibrėžti, kokius privalumus gali suteikti įvykiu kaupimo principu pagrįstos architektūros.

\begin{enumerate}

  \item \textbf{Audito žurnalas}. Reguliuojamose srityse (pvz. finansų industrija), daugelyje šalių valstybės nuostatai reikalauja kompanijų saugoti operacijų istoriją sistemoje. Pavyzdžiui, JAV reikalauja tarpininkų saugoti įrašus neperrašomu ir neištrinamu formatu \cite{US:StorageRules}. Įvykių kaupimo principas puikiai tinka šiam reikalavimui įgyvendinti, nes įvykių žurnalas pasižymi tik įrašymo funkcionalumu, o patys įvykiai yra nekintantys.

  \item \textbf{Derinimas}. Sukaupti įvykiai gali būti panaudoti analizei kaip sistema pasiekė vieną ar kitą būseną ir kurie įvykiai tai įtakojo. Įvykių kaupimo stiprioji pusė yra atsekamumas ir derinimo galymybės - įmanoma atsekti iš kur kilo sistemos klaida.

  \item \textbf{Išplečiamumas}. Faktas, jog į įvykių žurnalą galima tik įrašyti, yra naudingas išplečiamoms architektūroms. Gana dažnai įvykių kaupimo principu paremtose sistemose turima keletą duomenų modelio kopijų. Norint užtikrinti darną šios kopijos turi būti sinchronizuotos. Tokiose sistemose vienintelis būdas suvienodinti duomenų modelio kopijas yra įrašant įvykius. Manoma, jog toks sprendimas turi mažiau blokavimų ir lengviau išplečiamas skaitymui negu architektūra atnaujinant duomenų modelį \cite{GetEventStore:Basics}.

  \item \textbf{Informacinė nauda}.
 Visos buvusios sistemos būsenos gali būti tiek atkurtos, tiek gaunamos užklausų pagalba. Tai gali atnešti papildomos naudos sistemoms, kuriose kliento elgesys yra svarbus. Tokiose sistemose dažniausiai nežinoma kokio tyrimo reikės iš verslo pusės. Pavyzdžiui, elektroninės komercijos parduotuvėje gali būti naudinga gauti ir nagrinėti visus produktus, kurie kada nors buvo išimti iš apsipirkimo krepšelio. Įvykių kaupimo sistemose, toks uždavinys gali būti lengvai įgyvendintas.

\end{enumerate}

\subsection{Panaudojimo atvejai}

Internete galima rasti straipsnių aprašančių kaip įvykių kaupimo principas buvo panaudotas tikrose sistemose. Galima pažymėti dvi išsamias dokumentacijas: finansinės prekybos platforma LMAX \cite{Fowler:LMAX} ir Microsoft Windows Azure komandos analizė \cite{Betts:2013:ECE:2509680}. Taipogi yra sukurti keli pragramavimo karkasai paremti įvykių kaupimu: Akka\footnote{http://akka.io/} ir Event Store\footnote{http://docs.geteventstore.com/} bei Red Bull Media House kompanijos programavimo priemonių rinkinys Eventuate\footnote{https://rbmhtechnology.github.io/eventuate/}.

\subsection{Komandų-užklausų atskyrimo principas (angl. Command Query Separation (CQS))}

Komandų-užklausų atskyrimo principą pirmą kartą aprašė Bertrand Meyer \cite{Meyer:1988:OSC:534929} norėdamas patobulinti šalutinių efektų apdorojimą kuriant programą ar projektuojant API. Pagrindinė idėja yra atskirti prieigą prie objektų į:

\begin{enumerate}

  \item \textbf{Užklausas}, kurios grąžina informaciją,

  \item \textbf{Komandas}, kurios keičia būseną (Paveikslėlis \ref{img:cqs}.a).

\end{enumerate}

Užklausos neturėtų sukelti pašalinių efektų. Kitais žodžiais tariant - klausimas neturėtų pakeisti atsakymo.

\begin{figure}[H]
    \centering
    \includegraphics[scale=0.6]{img/cqs}
    \caption{CQS ir CQRS palyginimas}
    \label{img:cqs}
\end{figure}

\subsection{Komandų-užklausų atsakomybių atskyrimas (angl. Command Query Responsibility Segregation (CQRS))}

Komandų užklausų atsakomybių atskyrimas yra projektavimo šablonas, kurį pirma kartą aprašė Greg Young \cite{Young:CQRS2013}. CQRS pažymi, jog skirtingi modeliai (pavyzdžiui komponentas ar objektas) yra naudojami vykdyti komandas ir įvykdyti užklausas (Paveikslėlis \ref{img:cqs}.b). CQS atskiria atsakomybes kodo lygmenyje rašant užklausas ir komandas kaip atskirus metodus. Tuo tarpu CQRS atskiria objektus į dvi rūšis - skaitymo ir rašymo. Kitais žodžiais Greg Young apibūdina tai kaip: ``CQRS paprasčiausiai yra dviejų objektų sukūrimas ten, kur anksčiau buvo tik vienas'' \cite{Young:CQRS2010}.

Architektūros kūrimas, paremtas CQRS projektavimo šablonu dažnai veda prie ilgainės darnos (angl. eventual consistency) elgesio sistemoje. Kadangi užklausų ir komandų modeliai yra skirtingi komponentai, jie nebūtinai sinchronizuoti. Jeigu jie silpnai sujungti (angl. loose coupling), užklausų modelis nebūtinai atvaizduoja tuos pačius duomenis kaip komandų modelis tam tikru laiko momentu. Sąvybė, jog užklausų modelis gali gražinti pasenusią informaciją, kuri tam tikrame laiko taške ateityje ilgainiui tampa tiksli arba darni, vadinama ilgainė darna. Kai užklausų modeliai yra fiziškai atskirti nuo komandų modelių, ilgainė darna tampa ypatingai svarbi išsisklaidžiusioje sistemoje. Brewer CAP teorema  \cite{Fox:1999:HYS:822076.822436}

\subsection{Įvykių kaupimas ir CQRS}

\subsection{DDD}

